\chapter{جمع‌بندي و نتيجه‌گيری و پیشنهادها}
%%%%%%%%%%%%%%%%%%%%%%%%%%%%%%%%%%%%%%%%%%%
\section{جمع‌بندي و نتيجه‌گيری}

در این پروژه سعی شد بخشی از یک سیستم پیش بینی طول عمر تجهیزات صنعتی طبق \cite{jung2017vibration} و بر اساس داده‌های ارتعاشی پیاده‌سازی شود.

در فصل اول سعی شد مقدمه‌ای برای نگهداری و معرفی روش‌های مختلف نگهداری تجهیزات در محل‌‌های صنعتی امروزی بیان شود.

در فصل دوم اصول و تجهیزات مورد استفاده را توضیح دادیم و در مورد حسگر ارتعاش \lr{MEMS}، آردوینو نانو بعنوان گره پایانی، پروتکل زیگبی و ماژول ایکس‌بی، رزبری‌پای و استفاده از آن بعنوان دروازه، \lr{InfluxDB} بعنوان پایگاه داده سری زمانی، کتابخانه نامپای و پیش‌پردازش و چارچوب ویو برای توسعه برنامه وب صحبت کردیم.

در فصل سوم نحوه پیاده‌سازی سیستم را توضیح داده و یک نمای کلی از نحوه عملکرد سیستم ارائه دادیم. سپس کتابخانه‌ها، توابع، فایل‌ها و هدف آنها را در این پروژه توضیح دادیم. در پایان نیز چند تصویر از نتیجه برنامه وب نشان دادیم.

در فصل چهارم در مورد چالش‌ها و محدودیت‌هایی که در اجرای پروژه در بخش نرم‌افزار و سخت‌افزار با آن مواجه بودیم، صحبت کردیم.

\section{پیشنهادها}

این پروژه می‌تواند برای پیش‌بینی دقیق‌تر و بهینه‌تر بهبود یابد. در این بخش پیشنهادهای خود برای بهبود پروژه را توضیح می‌دهیم.

\subsection{حل محدودیت‌های سخت‌افزاری}

ما می‌توانیم از سایر بوردهای آردوینو که حافظه بیشتری دارند برای دستیابی به فرکانس‌های نمونه‌برداری بالاتر استفاده کنیم، اما باید مراقب مصرف انرژی و هزینه نیز باشیم، زیرا این بوردها با باتری و در تعداذ زیاد بعنوان گره‌های انتهایی کار خواهند کرد.

\subsection{پیاده‌سازی ارسال اعلان و برنامه تلفن همراه}
پیاده سازی پوش نوتیفیکیشن و اپلیکیشن موبایل

برای بهبود تجربه کاربری و اطلاع کاربران از هرگونه خرابی در تجهیزات، می‌توان با پیاده‌سازی نسخه تلفن همراه برنامه وب و ویژگی ارسال اعلان هشدار به کاربران را آسان‌تر کرد.

\subsection{تکمیل کتابخانه ایکس‌بی}

نسخه ما از کتابخانه فقط برای یک نوع بسته پیاده‌سازی شده است. با پیاده‌سازی انواع دیگر فریم‌ها، اجرای سیاست کنترل خطای دیگری علاوه بر زیگبی ممکن می‌شود. به این ترتیب می‌توانیم کنترل و بازیابی خطای لایه برنامه را نیز پیاده‌سازی کنیم.

\subsection{استفاده از سایر معیارهای محیطی}

اگرچه دما و رطوبت معیارهای اندازه‌گیری دقیقی برای کاربرد ما نیستند و به محیط اطراف خود وابسته‌اند، استفاده از آنها همراه با لرزش می‌تواند دقت پیش‌بینی‌ها را بهبود دهد.
