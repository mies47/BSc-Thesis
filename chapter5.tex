\chapter{جمع‌بندي و نتيجه‌گيری و پیشنهادها}
%%%%%%%%%%%%%%%%%%%%%%%%%%%%%%%%%%%%%%%%%%%
\section{جمع‌بندي و نتيجه‌گيری}

در این پروژه سعی شد بخشی از یک سیستم پیش بینی طول عمر تجهیزات صنعتی طبق \cite{jung2017vibration} و بر اساس داده‌های ارتعاشی پیاده‌سازی شود.

در فصل اول سعی شد مقدمه‌ای برای نگهداری و معرفی روش‌های مختلف نگهداری تجهیزات در محل‌‌های صنعتی امروزی بیان شود.

در فصل دوم اصول و تجهیزات مورد استفاده را توضیح دادیم و در مورد حسگر ارتعاش \lr{MEMS}، آردوینو نانو بعنوان گره پایانی، پروتکل زیگبی و ماژول ایکس‌بی، رزبری‌پای و استفاده از آن بعنوان دروازه، \lr{InfluxDB} بعنوان پایگاه داده سری زمانی، کتابخانه نامپای و پیش‌پردازش و چارچوب ویو برای توسعه برنامه وب صحبت کردیم.

در فصل سوم نحوه پیاده‌سازی سیستم را توضیح داده و یک نمای کلی از نحوه عملکرد سیستم ارائه دادیم. سپس کتابخانه‌ها، توابع، فایل‌ها و هدف آنها را در این پروژه توضیح دادیم. در پایان نیز چند تصویر از نتیجه برنامه وب نشان دادیم.

در فصل چهارم در مورد چالش‌ها و محدودیت‌هایی که در اجرای پروژه در بخش نرم‌افزار و سخت‌افزار با آن مواجه بودیم، صحبت کردیم.

در این پروژه با چالش‌ها و محدودیت‌هایی نیز مواجه شدیم که عبارتند از:

\begin{itemize}
\item حافظه کم در بورد آردوینو: بورد آردوینو نانو حافظه بسیار محدودی دارد و همین امر دستیابی به فرکانس‌های نمونه‌برداری بالاتر را غیرممکن می‌کرد، زیرا نمونه‌ها باید در حافظه ذخیره می‌شدند و سپس به دروازه ارسال می‌شد. از آنجایی که ما منابع لازم را نداشتیم، نتوانستیم به فرکانس مورد نظر دست یابیم، اما فرکانس نمونه‌برداری بعنوان یک ثابت در برنامه گره انتهایی تعریف شده است و در صورت وجود بورد مناسب قابل تغییر است.
\item محدودیت اندازه قاب‌های زیگبی: ماژول‌های ایکس‌بی برای ارسال بسته‌های با حجم بیش از ۲۵۶ بایت محدودیت دارند. برای انتقال هر بسته بزرگتر از آن، قطعه قطعه‌سازی لازم بود. اما ما با آزمایش مشاهده کردیم که باید زمانی بین ارسال هر قطعه در نظر گرفت. در غیر این صورت، برخی از بسته‌ها گاهی گم می‌شوند و حتی با سیاست ارسال مجدد پیش‌فرض ایکس‌بی نیز دریافت نمی‌شوند. این مسئله نیازی جدید برای تعریف بازه زمانی مناسب در سمت گیرنده برای فعال‌ماندن و تشخیص قطعات مختلف ایجاد کرد.
\item پیاده‌سازی کتابخانه ایکس‌بی: همانطور که قبلا ذکر کردیم، کتابخانه‌های موجود برای استفاده از ماژول‌های ایکس‌بی مشکلات فراوانی داشتند. نوشتن یک کتابخانه جدید حتی برای استفاده محدود در این زمینه چالشی واقعی بود زیرا باید در سطوح پایین و با عملیات‌های بیتی پیاده‌سازی می‌شد.
\item اشکال‌زدایی\LTRfootnote{Debugging}: ماژول ایکس‌بی به درگاه سریال آردوینو متصل است و درگاه \lr{USB} روی آردوینو که برای تغذیه یا برنامه‌ریزی استفاده می‌شود نیز به همان درگاه متصل است. بنابراین استفاده همزمان از مانیتور سریال\LTRfootnote{Serial Monitor} و ماژول ایکس‌بی ممکن نبود و برای برنامه‌ریزی آردوینو، مجبور بودیم ماژول را از جدا کنیم.

\end{itemize}

\section{پیشنهادها}

این پروژه می‌تواند برای پیش‌بینی دقیق‌تر و بهینه‌تر بهبود یابد. این پیشنهادها عبارتند از:

\begin{itemize}
\item حل محدودیت‌های سخت‌افزاری: ما می‌توانیم از سایر بوردهای آردوینو که حافظه بیشتری دارند برای دستیابی به فرکانس‌های نمونه‌برداری بالاتر استفاده کنیم، اما باید مراقب مصرف انرژی و هزینه نیز باشیم، زیرا این بوردها با باتری و در تعداد زیاد بعنوان گره‌های انتهایی کار خواهند کرد.
\item پیاده‌سازی ارسال اعلان و برنامه تلفن همراه: برای بهبود تجربه کاربری و اطلاع کاربران از هرگونه خرابی در تجهیزات، می‌توان با پیاده‌سازی نسخه تلفن همراه برنامه وب و ویژگی ارسال اعلان هشدار به کاربران را آسان‌تر کرد.
\item تکمیل کتابخانه ایکس‌بی: نسخه ما از کتابخانه فقط برای یک نوع بسته پیاده‌سازی شده است. با پیاده‌سازی انواع دیگر فریم‌ها، اجرای سیاست کنترل خطای دیگری علاوه بر زیگبی ممکن می‌شود. به این ترتیب می‌توانیم کنترل و بازیابی خطای لایه برنامه را نیز پیاده‌سازی کنیم.
\item استفاده از سایر معیارهای محیطی: اگرچه دما و رطوبت معیارهای اندازه‌گیری دقیقی برای کاربرد ما نیستند و به محیط اطراف خود وابسته‌اند، استفاده از آنها همراه با لرزش می‌تواند دقت پیش‌بینی‌ها را بهبود دهد.
\end{itemize}