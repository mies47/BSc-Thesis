\chapter{چالش‌ها و محدودیت‌ها}

در این بخش در مورد چالش‌ها و محدودیت‌هایی که در طول این پروژه با آن مواجه بودیم صحبت خواهیم کرد. برخی از این چالش‌ها مربوط به سخت‌افزار و پروتکل‌ها و برخی دیگر مربوط به نرم‌افزار بودند که در ادامه به آنها می‌پردازیم.

\section{چالش‌های سخت‌افزاری}

همانطور که در بخش‌های قبلی ذکر شد، بورد آردوینو نانو حافظه بسیار محدودی دارد و همین امر دستیابی به فرکانس‌های نمونه‌برداری بالاتر را غیرممکن می‌کرد، زیرا نمونه‌ها باید در حافظه ذخیره می‌شدند و سپس به دروازه ارسال می‌شد. از آنجایی که ما منابع لازم را نداشتیم، نتوانستیم به فرکانس مورد نظر دست یابیم، اما فرکانس نمونه‌برداری بعنوان یک ثابت در برنامه گره انتهایی تعریف شده است و در صورت وجود بورد مناسب قابل تغییر است.



ماژول‌های ایکس‌بی برای ارسال بسته‌های با حجم بیش از ۲۵۶ بایت محدودیت دارند. برای انتقال هر بسته بزرگتر از آن، قطعه قطعه‌سازی لازم بود. اما ما با آزمایش مشاهده کردیم که باید زمانی بین ارسال هر قطعه در نظر گرفت. در غیر این صورت، برخی از بسته‌ها گاهی گم می‌شوند و حتی با سیاست ارسال مجدد پیش‌فرض ایکس‌بی نیز دریافت نمی‌شوند. این مسئله نیازی جدید برای تعریف بازه زمانی مناسب در سمت گیرنده برای فعال‌ماندن و تشخیص قطعات مختلف ایجاد کرد.


\section{چالش‌های نرم‌افزاری}

همانطور که قبلا ذکر کردیم، کتابخانه‌های موجود برای استفاده از ماژول‌های ایکس‌بی مشکلات فراوانی داشتند. نوشتن یک کتابخانه جدید حتی برای استفاده محدود در این زمینه چالشی واقعی بود زیرا باید در سطوح پایین و با عملیات‌های بیتی پیاده‌سازی می‌شد.


اشکال‌زدایی\LTRfootnote{Debugging} یکی دیگر از مسائل مهم بود. ماژول ایکس‌بی به درگاه سریال آردوینو متصل است و درگاه \lr{USB} روی آردوینو که برای تغذیه یا برنامه‌ریزی استفاده می‌شود نیز به همان درگاه متصل است. بنابراین استفاده همزمان از مانیتور سریال\LTRfootnote{Serial Monitor} و ماژول ایکس‌بی ممکن نبود و برای برنامه‌ریزی آردوینو، مجبور بودیم ماژول را از جدا کنیم.


همچنین ماژول ایکس‌بی ما بسیار قدیمی بود. برای برنامه‌ریزی خود ماژول، باید از سیستم‌افزار\LTRfootnote{Firmware} ساخته‌شده سازنده استفاده می‌کردیم و از آنجایی که ماژول بسیار قدیمی بود، یافتن و دریافت سیستم‌افزاری که با ماژول ما سازگار باشد دشوار بود.

\section{جمع‌بندی و نتیجه‌گیری}

در این بخش درباره چالش‌ها و محدودیت‌های مختلف سخت‌افزاری و نرم‌افزاری که هنگام پیاده‌سازی پروژه با آن مواجه شدیم صحبت کردیم.
