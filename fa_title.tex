%% -!TEX root = AUTthesis.tex
% در این فایل، عنوان پایان‌نامه، مشخصات خود، متن تقدیمی‌، ستایش، سپاس‌گزاری و چکیده پایان‌نامه را به فارسی، وارد کنید.
% توجه داشته باشید که جدول حاوی مشخصات پروژه/پایان‌نامه/رساله و همچنین، مشخصات داخل آن، به طور خودکار، درج می‌شود.
%%%%%%%%%%%%%%%%%%%%%%%%%%%%%%%%%%%%
% دانشکده، آموزشکده و یا پژوهشکده  خود را وارد کنید
\faculty{دانشکده مهندسی کامپیوتر}
% گرایش و گروه آموزشی خود را وارد کنید
%\department{گرایش ...}
% عنوان پایان‌نامه را وارد کنید
\fatitle{پیاده‌سازی سیستم نگهداری و تعمیرات پیش‌بینانه تجهیزات بر بستر اینترنت اشیاء مبتنی بر تحلیل لرزش}
% نام استاد(ان) راهنما را وارد کنید
\firstsupervisor{دکتر حمیدرضا زرندی}
%\secondsupervisor{استاد راهنمای دوم}
% نام استاد(دان) مشاور را وارد کنید. چنانچه استاد مشاور ندارید، دستور پایین را غیرفعال کنید.
%\firstadvisor{نام کامل استاد مشاور}
%\secondadvisor{استاد مشاور دوم}
% نام نویسنده را وارد کنید
\name{}
% نام خانوادگی نویسنده را وارد کنید
\surname{}
%%%%%%%%%%%%%%%%%%%%%%%%%%%%%%%%%%
\thesisdate{تیر ۱۴۰۲}

% چکیده پایان‌نامه را وارد کنید
\fa-abstract{
در سال‌های اخیر اینترنت اشیاء به یکی از داغ‌ترین موضوعات فناوری تبدیل شده‌است. کاربرد این فناوری در تمامی حوزه‌های زندگی انسان و همچنین پیشرفت‌های اخیر در حوزه‌های جمع‌آوری داده، شبکه و هوش مصنوعی باعث شده‌اند که اینترنت اشیاء مورد توجه محققان قرار گیرد. یکی از چالش‌های موجود در صنایع و کارخانه‌ها تعویض بهینه قطعات است. با توجه به نبود اطلاعات کافی برای تحلیل وضعیت قطعات، راه‌حل مناسب برای اطمینان از کارکرد خط تولید، استفاده از متخصصان جهت بازرسی از وضعیت تجهیزات یا تعویض آنها بدون توجه به وضعیت فعلی و صرفا طبق زمان‌بندی قبلی است. این راه‌حل‌ها علاوه بر نداشتن دقت لازم، هزینه و زمان زیادی بر صنایع تحمیل می‌کنند. در این پروژه ابتدا داده‌های لرزش تجهیزات با استفاده از سنسور \lr{MEMS} اندازه‌گیری شده، سپس در بازه زمانی مشخص با استفاده از پروتکل زیگبی برای دروازه فرستاده می‌شود. پس از تجمیع حجم معینی از داده‌ها اطلاعات برای سرور فرستاده می‌شود. اطلاعات و تخمین عمر باقیمانده تجهیزات پس از پردازش در سرور، در یک صفحه وب نمایش داده می‌شود. در ادامه به توصیف کامل اقدامات انجام‌شده، دلایل انتخاب تجهیزات مختلف و همینطور چالش‌هایی که در انجام پروژه مواجه شدیم پرداخته شده‌است. در انتها نیز مواردی را که سبب بهبود و توسعه بیشتر پروژه می‌شود، پیشنهاد داده‌ایم.
}


% کلمات کلیدی پایان‌نامه را وارد کنید
\keywords{اینترنت اشیاء، نگهداری پیش‌بینانه، اینترنت اشیاء صنعتی، نگهداری سیستم سایبری-فیزیکی}



\AUTtitle
%%%%%%%%%%%%%%%%%%%%%%%%%%%%%%%%%%
\vspace*{7cm}
\thispagestyle{empty}
\begin{center}
\includegraphics[height=5cm,width=12cm]{besm}
\end{center}