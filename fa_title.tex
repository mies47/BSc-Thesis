%% -!TEX root = AUTthesis.tex
% در این فایل، عنوان پایان‌نامه، مشخصات خود، متن تقدیمی‌، ستایش، سپاس‌گزاری و چکیده پایان‌نامه را به فارسی، وارد کنید.
% توجه داشته باشید که جدول حاوی مشخصات پروژه/پایان‌نامه/رساله و همچنین، مشخصات داخل آن، به طور خودکار، درج می‌شود.
%%%%%%%%%%%%%%%%%%%%%%%%%%%%%%%%%%%%
% دانشکده، آموزشکده و یا پژوهشکده  خود را وارد کنید
\faculty{دانشکده مهندسی کامپیوتر}
% گرایش و گروه آموزشی خود را وارد کنید
\department{گرایش معماری سیستم‌های کامپیوتری}
% عنوان پایان‌نامه را وارد کنید
\fatitle{پياده‌سازی سيستم لرزه‌نگار جهت نگهداری و تعميرات پيش‌بينانه تجهيزات بر بستر اينترنت اشياء}
% نام استاد(ان) راهنما را وارد کنید
\firstsupervisor{دکتر حمیدرضا زرندی}
%\secondsupervisor{استاد راهنمای دوم}
% نام استاد(دان) مشاور را وارد کنید. چنانچه استاد مشاور ندارید، دستور پایین را غیرفعال کنید.
%\firstadvisor{نام کامل استاد مشاور}
%\secondadvisor{استاد مشاور دوم}
% نام نویسنده را وارد کنید
\name{میلاد}
% نام خانوادگی نویسنده را وارد کنید
\surname{اسرافیلیان نجف‌آبادی}
%%%%%%%%%%%%%%%%%%%%%%%%%%%%%%%%%%
\thesisdate{تیر ۱۴۰۲}

% چکیده پایان‌نامه را وارد کنید
\fa-abstract{در سال‌های اخیر اینترنت اشیاء به یکی از داغ‌ترین موضوعات فناوری تبدیل شده‌است. کاربرد این فناوری در تمامی حوزه‌های زندگی انسان و همچنین پیشرفت‌های اخیر در حوزه‌های جمع‌آوری داده، شبکه و هوش مصنوعی باعث شده‌اند که اینترنت اشیاء مورد توجه محققان قرار گیرد. یکی از چالش‌های موجود در صنایع و کارخانه‌ها تعویض بهینه قطعات است. با توجه به نبود اطلاعات کافی برای تحلیل وضعیت قطعات، راه‌حل مناسب برای اطمینان از کارکرد خط تولید، استفاده از متخصصان جهت بازرسی از وضعیت تجهیزات یا تعویض آنها بدون توجه به وضعیت فعلی و صرفا طبق زمان‌بندی قبلی است. این راه‌حل‌ها علاوه بر نداشتن دقت لازم، هزینه و زمان زیادی بر صنایع تحمیل می‌کنند. در این پروژه قصد داریم که با استفاده از ترکیب اینترنت اشیاء و هوش مصنوعی، چارچوبی برای تحلیل داده‌های لرزش تجهیزات ارائه گردد تا با استفاده از آن بتوان عمر مفید باقیمانده قطعات را بخوبی پیش‌بینی کرد. همچنین با استفاده از درگاه مبتنی بر وب، نتایج حاصله به افراد متخصص نشان داده شود تا بتوانند برنامه‌ریزی دقیقی برای نگهداری و تعمیر تجهیزات ارائه کنند. با بکارگیری این رویکرد هزینه و وقت صرف‌شده برای تعویض قطعات به‌شکل چشمگیری کاهش می‌یابد.}


% کلمات کلیدی پایان‌نامه را وارد کنید
\keywords{اینترنت اشیاء، نگهداری پیش‌بینانه، اینترنت اشیاء صنعتی، نگهداری سیستم سایبری-فیزیکی}



\AUTtitle
%%%%%%%%%%%%%%%%%%%%%%%%%%%%%%%%%%
\vspace*{7cm}
\thispagestyle{empty}
\begin{center}
\includegraphics[height=5cm,width=12cm]{besm}
\end{center}